\documentclass[sigconf]{acmart}

\usepackage{booktabs} % For formal tables

\newcommand{\todo}[1]
  {{\scriptsize \textbf{\color{red} {#1}}}}


% Copyright
%\setcopyright{none}
%\setcopyright{acmcopyright}
%\setcopyright{acmlicensed}
\setcopyright{rightsretained}
%\setcopyright{usgov}
%\setcopyright{usgovmixed}
%\setcopyright{cagov}
%\setcopyright{cagovmixed}



% DOI
\acmDOI{10.475/123_4}

% ISBN
\acmISBN{123-4567-24-567/08/06}

%Conference
\acmConference[MSR'18]{Mining Software Repositories}{May 2018}{Gothenburg, Sweden}
\acmYear{2018}
\copyrightyear{2018}


\begin{document}
\title{Some Name About Debugging}
%\titlenote{Produces the permission block, and
%  copyright information}
%\subtitle{Extended Abstract}
%\subtitlenote{The full version of the author's guide is available as
%  \texttt{acmart.pdf} document}


\author{Mauricio Soto}
\affiliation{%
  \institution{Carnegie Mellon University}
  \city{Pittsburgh}
  \state{PA}
  \postcode{15213}
}
\email{mauriciosoto@cmu.edu}

\author{Chu-Pan Wong}
\affiliation{%
  \institution{Carnegie Mellon University}
  \city{Pittsburgh}
  \state{PA}
  \postcode{15213}
}
\email{chupanw@cs.cmu.edu}

\author{Jens Meinicke}
\affiliation{%
  \institution{University of Magdeburg}
  \city{Germany}
}
\email{meinicke@ovgu.de}

\author{Christian K\"{a}stner}
\affiliation{%
  \institution{Carnegie Mellon University}
  \city{Pittsburgh}
  \state{PA}
  \postcode{15213}
}
\email{kaestner@cs.cmu.edu}

\author{Claire Le Goues}
\affiliation{%
  \institution{Carnegie Mellon University}
  \city{Pittsburgh}
  \state{PA}
  \postcode{15213}
}
\email{clegoues@cs.cmu.edu}


% The default list of authors is too long for headers.
%\renewcommand{\shortauthors}{B. Trovato et al.}


\begin{abstract}
\todo{Insert abstract}
\end{abstract}

%
% The code below should be generated by the tool at
% http://dl.acm.org/ccs.cfm
% Please copy and paste the code instead of the example below.
%
\begin{CCSXML}
<ccs2012>
<concept>
<concept_id>10011007.10011074.10011111.10011113</concept_id>
<concept_desc>Software and its engineering~Software evolution</concept_desc>
<concept_significance>100</concept_significance>
</concept>
</ccs2012>
\end{CCSXML}

\ccsdesc[100]{Software and its engineering~Software evolution}


\keywords{Debugging, Multi-edit automatic program repair}


\maketitle

\section{Introduction}



\section{Gathering Corpus}
\subsection{Delimiting Debugging Regions}
The first step towards being able to analyze the debugging
process developers go through is to be able to tell when
developers are trying to fix a bug. A very straightforward 
way to know when a developer is debugging is by keeping
record of when the developer triggers the debugger. An important
disadvantage of this approach is that 
developers often will try to fix errors in their code without the
need to trigger the debugger, and by following this approach,
these cases would not
be considered into the analysis.

In this study we take a broader approach to include the most 
instances where a developer is trying to fix an error.
We delimit the debugging regions by starting at a point
in time $\delta$ where a \textit{TestRunEvent} is triggered and 
all test cases were ran (the execution of test cases
was not aborted), but one or more test cases failed.
The delimitation ends at a point in time $\epsilon $ where $ \epsilon > \delta$ 
and where all test cases were ran and all test cases passed.
Figure~\ref{demarcations} shows an example of a time line
with only the \textit{TestRunEvent's}. "P" represents a case
where all test cases passed and "F" represents a case where one
or more test cases failed. The \texttt{debugging areas} are shown by
the brackets above with the "DA" initials. 


It is important to notice that the last \textit{TestRunEvent} before $\delta$ 
is always an event were all test cases passed, meaning that the
developer was not actively looking to fix a bug pointed out by the
failing test cases in the test suite. It is also worth noticing
that between delimiting points $\delta$ and $\epsilon$ there can be 
several \textit{TestRunEvent's} with failing test cases, meaning
that the developer is actively debugging the error pointed out
by the failing test cases, but still hasn't been able to make
all test cases pass.

\begin{figure}[h]
\label{demarcations}
\caption{Example of the process used to obtain the demarcation of
debugging areas. "P" 
represents an event where the developer ran a test suite and all test cases
passed, "F" is used when one or more tests failed. The horizontal brackets
above show the different debugging areas ("DA").}
\centering
\includegraphics[width=0.5\textwidth]{images/demarcations.png}
\end{figure}

We were able to obtain 634 debugging areas in the dataset 
provided by the MSR Challenge~\cite{msr18challenge}.
Within these 634 debugging areas, there are 1,251,334
different events, from which 1,748 are \textit{EditEvent's}
that have associated a not-unknown Contexts, and therefore 
contain a valid Simplified Syntax Tree. These will be 
our primary subject of analysis.

\subsection{Simplified Syntax Tree Differencing}
Once we have two consecutive \textit{EditEvent's} within
a debugging region, we procede to analyze the differences
between their two associated Simplified Syntax Trees (SST).
These are the modifications the developer performed
while debugging.
To perform the difference between trees we use the 
state-of-the-art tree differencing
tool APTED~\cite{Pawlik16Apted}.
We run recursively through each of the statements in the 
SST's creating a representation that can be understood by
APTED. This tool returns a list of mappings between
the SST of the previous \textit{EditEvent} and the SST
of the following \textit{EditEvent}. This mapping
accounts for node insertitons, deletions, and renaming.
Once we know which nodes were modified from one \textit{EditEvent}
to the following, we record the types of the modified statements.

\section{Evaluation}

\section{Threats to Validity}

\section{Related Work}

\section{Conclusions}



\begin{acks}
 This section will be added for the camera-ready version.

\end{acks}


\bibliographystyle{ACM-Reference-Format}
\bibliography{acmart}

\end{document}
